% e. woollett
%  2nd version 2012
%  1st version july 08 - july 09
% file nint.tex this is new ch. 8
% things to add: 
%Also, there are functions float_approx_equal and bfloat_approx_equal.
%The default for float_approx_equal_tolerance is about 8 * unit roundoff.
%
%(%i1) float_approx_equal_tolerance;
%(%o1) 8.881784197001254E-16
%
%(%i2) float_approx_equal(float(2/3), 0.66666666666666666666666666666);
%(%o2) true
%
%(%i4) float_approx_equal(float(2/3), 0.66666666666);
%(%o4) false
%
%Source code comment:
%
%;; Return true if $float converts both a and b to floats and
%;; |a - b| <= float_approx_equal_tolerance * min(|a|, |b|).
%;; In all other cases, return false.
%
%
%Barton
% edit with Notepad++, then load into LED for latexing
\documentclass[12pt]{article}
\usepackage[dvips,top=1.5cm,left=1.5cm,right=1.5cm,foot=1cm,bottom=1.5cm]{geometry}
\usepackage{times,amsmath,amsbsy,graphicx,fancyvrb,url}
\usepackage[usenames]{color}
%\definecolor{MyDarkBlue}{rgb}{0,0.08,0.45}
\definecolor{mdb}{rgb}{0.1,0,0.55}
\newcommand{\tcdb}{\textcolor{mdb}}
\newcommand{\tcbr}{\textcolor{BrickRed}}
\newcommand{\tcb}{\textcolor{blue}}
\newcommand{\tcr}{\textcolor{red}}
\urldef\tedhome\url{ http://www.csulb.edu/~woollett/  }
\urldef\tedmail\url{ woollett@charter.net}
%1.  this is for maxima code: red framed bold, footnotesize 
\DefineVerbatimEnvironment%
   {myVerbatim}%
   {Verbatim}%
   {fontfamily=courier,fontseries=b,fontsize=\footnotesize ,frame=single,rulecolor=\color{BrickRed}}
\DefineVerbatimEnvironment%
   {myVerbatim1}%
   {Verbatim}%
   {fontfamily=courier,fontseries=b,fontsize=\scriptsize ,frame=single,rulecolor=\color{BrickRed}}
%2.  this is for blue framed bold 
\DefineVerbatimEnvironment%
   {myVerbatim2}%
   {Verbatim}%
   {fontfamily=courier,fontseries=b,frame=single,rulecolor=\color{blue}}
\DefineVerbatimEnvironment%
   {myVerbatim2s}%
   {Verbatim}%
   {fontfamily=courier,fontseries=b,fontsize=\small,frame=single,rulecolor=\color{blue}}
\DefineVerbatimEnvironment%
   {myVerbatim2f}%
   {Verbatim}%
   {fontfamily=courier,fontseries=b,fontsize=\footnotesize,frame=single,rulecolor=\color{blue}}
% 3.  this is for black framed  bold
\DefineVerbatimEnvironment%
   {myVerbatim3}%
   {Verbatim}%
   {fontfamily= courier, fontseries=b, frame=single}
% 4.  this is for no frame bold
\DefineVerbatimEnvironment%
   {myVerbatim4}%
   {Verbatim}%
   {fontfamily=courier, fontseries=b}
% 6.  for defaults use usual verbatim
\newcommand{\mv}{\Verb[fontfamily=courier,fontseries=b]}
\newcommand{\mvs}{\Verb[fontfamily=courier,fontseries=b,fontsize=\small]}
\newcommand{\mvf}{\Verb[fontfamily=courier,fontseries=b,fontsize=\footnotesize]}

\renewcommand{\thefootnote}{\ensuremath{\fnsymbol{footnote}}}
%%%%%%%%%%%%%%%%%%%%%%%%%%%%%%%%%%%%%%%%%%%%%%%%%%%%%%%%%%%%%%%%%%%%%%%%
%   title page
%%%%%%%%%%%%%%%%%%%%%%%%%%%%%%%%%%%%%%%%%%%%%%%%%%%%%%%%%%%%%%%%%%%%%%

\title{  Maxima by Example:\\ Ch.8: Numerical Integration 
            \thanks{This version uses \textbf{Maxima 5.28.0}. This is a live
            document. Check \; \textbf{ \tedhome } \; for the latest version of these notes. Send comments and
			 suggestions to \textbf{\tedmail} } }


\author{ Edwin L. Woollett}
\date{\today}
%%%%%%%%%%%%%%%%%%%%%%%%%%%%%%%%%%%%%%%%%
%          document
%%%%%%%%%%%%%%%%%%%%%%%%%%%%%%%%%%%%%%%%%%
\begin{document}
%\footnotesize
%\small
\maketitle
\tableofcontents
\numberwithin{equation}{section}
\newpage
\subsubsection*{Preface}
\begin{myVerbatim2} 
COPYING AND DISTRIBUTION POLICY    
This document is part of a series of notes titled
"Maxima by Example" and is made available
via the author's webpage http://www.csulb.edu/~woollett/
to aid new users of the Maxima computer algebra system.	
	
NON-PROFIT PRINTING AND DISTRIBUTION IS PERMITTED.
	
You may make copies of this document and distribute them
to others as long as you charge no more than the costs of printing.	

These notes (with some modifications) will be published in book form
eventually in an arrangement which will continue
to allow unlimited free download of the pdf files as well as the option
of ordering a low cost paperbound version of these notes.
\end{myVerbatim2}	
\smallskip
\noindent \tcbr{Feedback from readers is the best way for this series of notes
  to become more helpful to new users of Maxima}.
\tcdb{\emph{All} comments and suggestions for improvements will be appreciated and
  carefully considered}.
\smallskip
\begin{myVerbatim2s}
The Maxima session transcripts were generated using the Xmaxima 
graphics interface on a Windows XP computer, and copied into 
a fancy verbatim environment in a latex file which uses the
fancyvrb and color packages.
\end{myVerbatim2s}  
\smallskip
\begin{myVerbatim2s}
  Maxima, a Computer Algebra System.
  Some numerical results depend on the Lisp version used.
  This chapter uses Version 5.28.0 (2012) using Lisp GNU 
  Common Lisp (GCL) GCL 2.6.8 (aka GCL).
  http://maxima.sourceforge.net/
\end{myVerbatim2s}
The author would like to thank the Maxima developers for their friendly help via 
  the Maxima mailing list. Richard Fateman provided early encouragement and
  advice. Raymond Toy, Stavros Macrakis, and Barton Willis have each
  contributed importantly to the evolution of the code in the \mv|nint| package.\\
  
\noindent The present \mv|nint| and \mv|apnint| packages are a work in progress, partially dictated by
the progress in Maxima special function float and bfloat values. The new Ch. 8 functions
\mv|apnint| and \mv|apquad| are the first step in providing quadrature results which
attempt to satisfy user requested precision (for one dimensional quadrature).
\newpage
\setcounter{section}{8}
\pagestyle{headings}
\subsection{  Introduction}
Chapter 8 on numerical integration (quadrature) is divided into eight sections.\\

\noindent The syntax of the functions \mv|nint| and \mv|quad| is exactly the
same (\mv|quad| is used to force use of Quadpack code) and this syntax is 
introduced with a few examples.
These functions are available once you load our \mv|nint| package, and they provide an interface to
Maxima's \mv|integrate| function and the Quadpack functions for numerical quadrature in one
and two dimensions.\\

\noindent When you use \mv|nint|, an answer is first sought using \mv|integrate| (unless the
integrand is a multiple-valued function or otherwise potentially dangerous) and 
only if \mv|integrate| does not succeed is Quadpack resorted to.\\

\noindent The new package functions, \mv|apnint| and \mv|apquad|,
 available if you both load the \mv|nint.mac| package and then load
 the \mv|apnint.mac| package (see examples below), allow the user to
  request quadrature results with more digits of precision (for one-dimensional
  quadrature)  than is returned by  \mv|nint| and \mv|quad|.
Several examples of the use of these functions is given.\\

\noindent We then discuss the most useful Quadpack functions for direct use, their syntax and
examples of direct use.\\

\noindent We then discuss the less often used Quadpack functions which may occasionally
be useful.\\

\noindent Finally we present a ``decision tree'' for
  one dimensional quadrature using the Quadpack functions, first for a finite
  integration interval, and then for a non-finite integration interval.\\

\noindent In Chapter 9, we will discuss bigfloats and details of using Maxima for high
  precision quadrature, parts of which are incorporated into the new Ch. 8 functions
  \mv|apnint| and \mv|apquad|.
Chapter 10 covers the calculation of Fourier series coefficients and both 
  Fourier transform and Laplace transform type integrals.  
Chapter 11  presents tools for the use of fast Fourier
  transforms with examples of use.\\
  
\noindent If you are new to Maxima, check out Ch. 1 of Maxima by Example, and
  also take a look at the web page:
\begin{myVerbatim2}
  http://mathscitech.org/articles/maxima
\end{myVerbatim2}
as well as the ``10 minute tutorial'':
\begin{myVerbatim2}
http://math-blog.com/2007/06/04/
         a-10-minute-tutorial-for-solving-math-problems-with-maxima/
\end{myVerbatim2}
and the Harvard Math web page:
\begin{myVerbatim2}
http://www.math.harvard.edu/computing/maxima/
\end{myVerbatim2}
\subsection{Using \textbf{nint} and \textbf{quad} for One or Two Dimensional Quadrature}
\subsubsection{Loading the \textbf{nint} Package}
Maxima will be ready to use the \mv|nint| package once you have issued a
  \mv|load| command such as \mv|load(nint)| (if you have set up your \mv|maxima.init|
  file as suggested in Ch. 1), or, using a complete path, as in\\
  \mv|load("c:/work2/nint.mac")|.
\begin{myVerbatim}
(%i1) load(nint);
(%o1) "c:/work2/nint.mac"
\end{myVerbatim}
The files that are loaded are \mv|nint.mac|, \mv|nint.lisp|, \mv|quad_util.mac|,
  \mv|mydefint.mac|, \mv|quad1d.mac|, and \mv|quad2d.mac|. 
Global parameters which are set during this load are:
  \mv|display2d:false|,\\
  \mv|ratprint:false|, \mv|domain:complex|,  and \mv|lognegint:true|.
\subsubsection{\textbf{nint} and \textbf{quad} Syntax}
Here is a summary of the  syntax for either \mv|nint| or \mv|quad|.\\

\noindent We use \mv|var| for the variable of integration, and \mv|a| and \mv|b|
  for the range limits in the 1d examples here (naturally, you can use
  any symbol for the variable of integration, as long as the supplied
  expression \mv|expr| is a function of that variable.\\
  
\noindent  For the 2d syntax examples, we assume \mv|expr| is a
  function of \mv|x| and \mv|y|, with \mv|x1| and \mv|x2| the
  range limits for the \mv|x| variable, and with \mv|y1| and
  \mv|y2| the range limits for the \mv|y| variable.\\
  
\noindent Again, you can use any symbols for the variables of integration.\\
  
\noindent The range limit parameters are either real numbers or
\mv|inf| or \mv|minf|.
\subsubsection*{1D INTEGRAL SYNTAX}
To first try integrate with a fallback to quadpack:
\begin{myVerbatim2}
        nint ([expr],var,a, b [options])
\end{myVerbatim2}
 or, to force use of quadpack:
\begin{myVerbatim2}
        quad ([expr],var,a, b [options])
\end{myVerbatim2}
The options: \mv|strong_osc|, \mv|singular|, \mv|principal_val(v0)|,
         \mv|points(x1,x2,...)|, \mv|real|,\\
		 \mv|imaginary|, and \mv|complex| are
         recognised for one dimensional integrals and are used
		 to decide what quadpack routine to use and how.\\
		 
\noindent Before calling the quadpack routines,
		\mv|nint| (or \mv|quad|) expends some effort in trying to confirm that
        the integrand is either real or pure imaginary. If
        \mv|nint| (or \mv|quad|) cannot assume the integrand is either real or
        pure imaginary (using a series of tests), \mv|nint| (or \mv|quad|)
        proceeds to work with the real and imaginary parts
        separately. Although it is never necessary to use the
		options \mv|real|, \mv|imaginary|, or \mv|complex|, for a complicated
		integrand some time may be saved if the relevant option
		is included (thus avoiding the time for the tests). Of course,
		if you include the option \mv|real|, you should be sure the 
		integrand evaluates to a real number over the whole domain of
		integration!\\
		
\noindent The option \mv|strong_osc| forces
        use of \mv|quad_qag| if integrate does not succeed.
        The \mv|strong_osc| keyword should only
        be used for a finite interval of integration.\\
		
\noindent The option \mv|singular| forces use of \mv|quad_qags| if
		integrate does not succeed; should only be used
		for a finite interval of integration.\\
        
\noindent The option \mv|principal_val(v0)| assumes that
        \mv|expr| has the form \mv|g(var)/(var - v0)|, and requests
        that the quadpack routine \mv|quad_qawc| be used for
        a numerical principal value evaluation (if
        integrate is not used or is not successful).\\
        
\noindent The option \mv|points(x1,x2,..)| forces use of
        \mv|quad_qagp| if \mv|integrate| is not used
        or is not successful. \mv|quad_qagp| gives special
        treatment to the supplied points, where the
        integrand is assumed to have singular behavior of
        some type. The supplied points should be entirely
        inside the range of integration.		
\begin{myVerbatim2}
   1d examples ( in each you can use quad instead of nint,
		    which forces use of quadpack routines ):
        
        nint (expr,var,a,b)  
        nint (expr,var,a,b,real)
        nint (expr,var,a,b,imaginary)
        nint (expr,var,a,b,complex)

        nint (expr,var,a,b,strong_osc)
        nint (expr,var,a,b,strong_osc,real)
        nint (expr,var,a,b,strong_osc,imaginary)
        nint (expr,var,a,b,strong_osc,complex)

        nint (expr,var,a,b,singular)
        nint (expr,var,a,b,singular,real)
        nint (expr,var,a,b,singular,imaginary)
        nint (expr,var,a,b,singular,complex)
        
        nint (expr,var,a,b,principal_val(v0))
		   where v0 evaluates to a number and
		    a < v0 < b.
        nint (expr,var,a,b,principal_val(v0),real)
        nint (expr,var,a,b,principal_val(v0),imaginary)
        nint (expr,var,a,b,principal_val(v0),complex)
		
        nint (expr,var,a,b, points(x1,x2,...))
           where x1, x2, etc., evaluate to numbers and 
		   a < x1 < b, etc.
        nint (expr,var,a,b, points(x1,x2,...),real)
        nint (expr,var,a,b, points(x1,x2,...),imaginary)
        nint (expr,var,a,b, points(x1,x2,...),complex)
\end{myVerbatim2}
\newpage
\subsubsection*{2D INTEGRAL SYNTAX}
We use a 2d notation similar to Mathematica's.
\mv| nint(f,[x,x1,x2],[y,y1,y2])| is the approximate numerical value of
the double integral \mv|integrate( integrate (f,y,y1,y2), x,x1,x2)|.
The 2d quadrature syntax is then:
\begin{myVerbatim2}
    nint (expr,[x,x1,x2 [x-option]],[y,y1,y2 [y-option]] [gen-option]) 
\end{myVerbatim2}
		  which includes a try using integrate, or
\begin{myVerbatim2}
    quad (expr,[x,x1,x2 [x-option]],[y,y1,y2 [y-option]] [gen-option]) 
\end{myVerbatim2}
which uses only the quadpack routines.
%\newpage
For example (in each, you can replace
  \mv|nint| by \mv|quad|):
\begin{myVerbatim2}
        nint (expr,[x,x1,x2],[y,y1,y2]) 				
		
        nint (expr,[x,x1,x2],[y,y1,y2],real)
       
        nint (expr,[x,x1,x2],[y,y1,y2,strong_osc])        
		
        nint (expr,[x,x1,x2,strong_osc],[y,y1,y2],imaginary)
        
        nint (expr,[x,x1,x2],[y,y1,y2,principal_val(y0)])        
        
        nint (expr,[x,x1,x2],[y,y1,y2,points(ya,yb,...)])        
\end{myVerbatim2}		
\subsubsection*{GENERAL REMARKS}		
\mv|nint| calls \mv|nint1d| for a 1d integral or \mv|nint2d| for a 2d integral.
     \mv|nint1d| and \mv|nint2d| are responsible for making sure the work detail
	 lists \mv|nargL| and \mv|noutL| are constructed and available as global
     lists if \mv|integrate| is used and is successful.\\
     
\noindent Otherwise these lists are constructed by the
     code in \mv|quad1d| or \mv|quad2d|, which are called if the
	 word \mv|nint| is replaced by \mv|quad|, or which are called
	 if the call to \mv|integrate| (by \mv|nint|) is not sucessful.\\
     
\noindent \mv|nargL| is a global list of method and args input,
      \mv|noutL| is a global list of method and either
      \mv|integrate| or quadpack total output.\\
      
\noindent If \mv|method| is set to \mv|true|, \mv|nint| or \mv|quad| will print out
      the method used during the work.\\
      
\noindent If \mv|debug| is set to \mv|true|, many details of progress
      will be printed to the console screen.
\subsubsection{1D Quadrature Using \textbf{mdefint} and \textbf{ndefint} }
You might want to try using (i.e., force Maxima to use) \mv|integrate|,
 even though \mv|nint| avoids using \mv|integrate| for a particular integrand.\\
  
\noindent For 1d integrals, the package provides the function \mv|mdefint| 
 which uses \mv|integrate| (in a careful way)
  to try to get a symbolic answer. You can then use \mv|cfloat| to reduce the answer to
  a floating point value. \mv|cfloat| is a package function which combines \mv|rectform|,
  \mv|float|, and \mv|expand| in one function, with the definition:
\begin{myVerbatim2}
      cfloat(expr):= (expand(float(rectform(expr))))$
\end{myVerbatim2}
However, if the symbolic answer is the difference of two almost equal values, \mv|cfloat| can
  return an incorrect numerical value. A more careful reduction to a numerical value
  occurs with the use of the package function \mv|fbfloat (expr, ndigits)| 
  which uses bigfloat methods to  reduce \mv|expr| to a floating point
  value using the requested precision.\\
  
\noindent The package function \mv|ndefint| combines \mv|mdefint| with \mv|fbfloat| 
 and requests 32 digit accuracy.
%\newpage
\noindent Some examples of the use of \mv|mdefint| and \mv|ndefint| for 1d
 quadrature (see below for 2d examples):
 \begin{myVerbatim}
(%i1) load(nint);
(%o1) "c:/work2/nint.mac"
(%i2) mdefint(log(1/x)/sqrt(x),x,0,1);
(%o2) 4
(%i3) ndefint(log(1/x)/sqrt(x),x,0,1);
(%o3) 4.0
(%i4) mdefint(log(1/x)/sqrt(%i*x),x,0,1);
(%o4) -4*(-1)^(3/4)
(%i5) ndefint(log(1/x)/sqrt(%i*x),x,0,1);
(%o5) 2.82842712474619-2.82842712474619*%i
(%i6) mdefint(log(1/x)/sqrt(%i*x),x,0,1);
(%o6) -4*(-1)^(3/4)
(%i7) cfloat(%);
(%o7) 2.828427124746191-2.828427124746191*%i
(%i8) fbfloat(%o6,32);
(%o8) 2.82842712474619-2.82842712474619*%i
 \end{myVerbatim}
\subsubsection{1D Quadrature Examples Using \textbf{nint} and \textbf{quad} }
We continue with the same integrand used above. Rather than setting
  \mv|method| to \mv|true|, we can use the global parameter \mv|noutL|
  to see what actual method(s) were used and perhaps some details.
  Even if we set \mv|method| to  \mv|true|, we can always look at
  \mv|noutL|;  more details are given if quadpack was used.\\
  
\noindent If no method option is included in the arguments given
  to \mv|nint| or \mv|quad|, and if quadpack is being used,
  \mv|quad1d| runs a competition between
   \mv|quad_qags| and \mv|quad_qag|, and selects the result with
  the least estimated error magnitude.  
\begin{myVerbatim}
(%i1) load(nint);
(%o1) "c:/work2/nint.mac"
(%i2) nint(log(1/x)/sqrt(%i*x),x,0,1);
(%o2) 2.82842712474623-2.828427124746158*%i
(%i3) noutL;
(%o3) [[qags,2.82842712474623,3.6770586575585185E-13,315,0],
       [qags,-2.828427124746158,1.1102230246251565E-13,315,0]]
(%i4) quad(log(1/x)/sqrt(%i*x),x,0,1);
(%o4) 2.82842712474623-2.828427124746158*%i
(%i5) noutL;
(%o5) [[qags,2.82842712474623,3.6770586575585185E-13,315,0],
       [qags,-2.828427124746158,1.1102230246251565E-13,315,0]]
(%i6) method:true$
(%i7) nint(log(1/x)/sqrt(%i*x),x,0,1);
   quad_qags 
   quad_qags 
(%o7) 2.82842712474623-2.828427124746158*%i
\end{myVerbatim}
\newpage
\begin{myVerbatim}
(%i8) quad(log(1/x)/sqrt(%i*x),x,0,1);
   quad_qags 
   quad_qags 
(%o8) 2.82842712474623-2.828427124746158*%i
(%i9) noutL;
(%o9) [[qags,2.82842712474623,3.6770586575585185E-13,315,0],
        [qags,-2.828427124746158,1.1102230246251565E-13,315,0]]
(%i10) nint(x^2,x,0,2);
 integrate 
(%o10) 2.666666666666667
(%i11) mdefint(x^2,x,0,2);
(%o11) 8/3
(%i12) cfloat(%);
(%o12) 2.666666666666667
\end{myVerbatim}
\subsubsection*{Non-Finite Range Examples}
\begin{myVerbatim}
(%i13) nint(1/(1+x^2),x,0,inf);
 integrate 
(%o13) 1.570796326794897
(%i14) nint(exp (%i*x^2),x,minf,inf);
 integrate 
(%o14) 1.2533141373155*%i+1.2533141373155
(%i15) quad(exp (%i*x^2),x,minf,inf);
 quad_qagi  
 realpart quad_qagi error =  integration failed to converge 
(%o15) false
\end{myVerbatim}
%\newpage
\noindent In the last example, \mv|integrate| is able to find the result, but
  quadpack fails. The symbolic answer is:
\begin{myVerbatim}
(%i16) mdefint(exp (%i*x^2),x,minf,inf);
(%o16) sqrt(%pi)*(%i/sqrt(2)+1/sqrt(2))
\end{myVerbatim}
\subsubsection*{Further Examples}
Because the integrand contains \mv|sqrt|, \mv|nint| bypasses \mv|integrate|
  here and calls \mv|quad1d|. However,\\
  \mv|integrate| can actually do this
  integral and returns the correct numerical value.
\begin{myVerbatim}
(%i17) nint(sin(x)/sqrt(x),x,0,5000,real,singular);
 quad_qags 
(%o17) 1.251128192999518
(%i18) ndefint(sin(x)/sqrt(x),x,0,5000);
(%o18) 1.251128192999533
\end{myVerbatim}
Here is a bessel function example in which we request the
  use of \mv|quad_qag|:
\begin{myVerbatim}
(%i19) nint(bessel_j(0,x)/(1+x),x,0,1,real,strong_osc);
 quad_qag 
(%o19) 0.64654342647716
\end{myVerbatim}
\newpage
Here is a principal value integral done by quadpack:
\begin{myVerbatim}
(%i20) quad((x-x^2)^(-1),x,-1/2,1/2,principal_val(0));
 quad_qawc 
 quad_qawc  
(%o20) 1.09861228866811
\end{myVerbatim}
However, \mv|integrate| can do this principal value integral, and issues the
  ``Principal Value'' printout to warn the user:
\begin{myVerbatim}
(%i21) ndefint((x-x^2)^(-1),x,-1/2,1/2);
Principal Value
(%o21) 1.09861228866811
\end{myVerbatim}
Here is a quadrature done using the quadpack routine \mv|quad_qagp| by including
  the \mv|points| option:
\begin{myVerbatim}
(%i22) quad(1/sqrt(sin(x)),x,0,10,points(%pi,2*%pi,3*%pi));
 quad_qagp 
  quad_qagp  
  quad_qagp  
(%o22) 10.48823021716681-6.769465521725387*%i 
\end{myVerbatim}
\mv|integrate| cannot find a value for this integral:
\begin{myVerbatim}
(%i23) ndefint(1/sqrt(sin(x)),x,0,10);
(%o23) false
(%i24) mdefint(1/sqrt(sin(x)),x,0,10);
(%o24) 'integrate(1/sqrt(sin(x1030868)),x1030868,0,10)
\end{myVerbatim}
\subsubsection{2D Quadrature Examples Using \textbf{nint} and \textbf{quad} }
The no-option version of 2D syntax also works with \mv|mdefint| and \mv|ndefint|.
\begin{myVerbatim}
(%i1) load(nint);
(%o1) "c:/work2/nint.mac"
(%i2) nint(1,[x,0,1],[y,0,1]);
(%o2) 1.0
(%i3) quad(1,[x,0,1],[y,0,1]);
(%o3) 1.0
(%i4) mdefint(1,[x,0,1],[y,0,1]);
(%o4) 1
(%i5) ndefint(1,[x,0,1],[y,0,1]);
(%o5) 1.0
\end{myVerbatim}
 and here are uses of the option \mv|real|:
\begin{myVerbatim}
(%i6) nint(1/sqrt(x+y),[x,0,1],[y,0,1],real);  
(%o6) 1.104569499660576
(%i7) nint(sin(x*y),[x,0,1],[y,0,1],real);  
(%o7) 0.23981174200056
\end{myVerbatim}
\newpage
A 2d example in which the presence of \mv|log| and also fractional powers causes
  \mv|integrate| to be bypassed in favor of quadpack (set \mv|method| to \mv|true|
  to see easily what method is being used here):
\begin{myVerbatim}
(%i8) method:true$
(%i9) nint(log(y)/x^(4/5),[x,0,1],[y,0,1]);
  qags2  
(%o9) -5.000000000000004
(%i10) time(%);
(%o10) [6.23]
(%i11) nint(log(y)/x^(4/5),[x,0,1],[y,0,1],real);
  qags2  
(%o11) -5.000000000000004
(%i12) time(%);
(%o12) [3.78]
(%i13) ndefint(log(y)/x^(4/5),[x,0,1],[y,0,1]);
(%o13) -5.0
(%i14) time(%);
(%o14) [0.02]
\end{myVerbatim}
This is a case in which including the option \mv|real| halves the time
  of the quadpack calculation.
This integrand-domain combination is correctly evaluated by \mv|mdefint| in a fraction
  of the time required by quadpack.
\begin{myVerbatim}
(%i15) mdefint(log(y)/x^(4/5),[x,0,1],[y,0,1]);
(%o15) -5
(%i16) time(%);
(%o16) [0.0]
\end{myVerbatim}
2d examples with non-finite limits:
\begin{myVerbatim}
(%i17) nint(exp(-abs(x) - abs(y)),[x,minf,inf],[y,minf,inf]);
 integrate 
(%o17) 4.0
(%i18) quad(exp(-abs(x) - abs(y)),[x,minf,inf],[y,minf,inf]);
 qagi(qagi(..)) 
(%o18) 3.999999999996149
(%i19) nint(sin(x*y)*exp(-x)*exp(-y),[x,0,inf],[y,0,inf],real);
 qagi(qagi(..)) 
(%o19) 0.34337796296064
\end{myVerbatim}
\subsubsection{The \textbf{nint} Package Sets \textbf{domain} to \textbf{complex}: Caution!}
When you load in the \mv|nint| package, \mv|domain| is set to \mv|complex|,
which can result in an incorrect symbolic integral result or an error
when you directly use \mv|integrate|.\\

\noindent Here is a 2d example, in which
\mv|integrate| gives the correct answer only if \mv|domain| is set to \mv|real|
(the normal default in the absence of the package \mv|nint.mac|).
(This is a Maxima bug.)\\

\noindent If you are only interested in the numerical value (for a particular value of
 the parameter \mv|b|), and use our package 
functions \mv|nint| or \mv|quad|, there is no problem, because the \mv|sqrt|
function in the integrand causes \mv|nint| to bypass \mv|integrate| automatically.
\begin{myVerbatim}
(%i1) load(nint);
(%o1) "c:/work2/nint.mac"
(%i2) domain;
(%o2) complex
\end{myVerbatim}
\newpage
\begin{myVerbatim}
(%i3) foo :((-%pi)/2+acos(y/b)+acos(x/b))/(2*%pi)$
(%i4) assume(b>0,x>0, x<b);
(%o4) [b > 0,x > 0,b > x]
(%i5) integrate(integrate(foo,y,0,sqrt(b^2 - x^2)),x,0,b);
expt: undefined: 0 to a negative exponent.
 -- an error. To debug this try: debugmode(true);
(%i6) domain:real;
(%o6) real
(%i7) integrate(integrate(foo,y,0,sqrt(b^2 - x^2)),x,0,b);
(%o7) b^2/(4*%pi)
\end{myVerbatim}
\subsubsection{Case: Symbolic Definite Integral the Difference of Two Almost Equal Numbers }
Here we compare the direct use of \textbf{integrate}, \textbf{nint}, and
  \textbf{quad} with an integrand whose definite integral is the difference
  of two almost equal numbers.\\
  
\noindent Use of \mv|cfloat| (an \mv|nint| package function which combines \mv|rectform|,
\mv|float|, and \mv|expand|) produces a wildly wrong numerical answer.
Use of the package function \mv|fbfloat|, and choice of 32 digit precision, 
uses bigfloat methods  to get an accurate answer, which is then 
converted back to 16 digit display inside the function \mv|fbfloat|.\\

\noindent The package function \mv|fchop| is used routinely in the package to ignore
  very small numbers.\\
  
\noindent The package function \mv|nint| tries \mv|integrate| first, and if
 successful then uses \mv|fbfloat| with 32 digit precision requested to obtain the numerical
 value from the symbolic definite integral, and then uses \mv|fchop|.\\

\noindent The package function \mv|quad| always uses \mv|fchop| on the result
returned by quadpack to round off the numerical  answer.
(But see below about turning off the chopping of very small numbers.)\\

\noindent It is clear, from the nature of the integrand, that the answer must be a
real number.
\begin{myVerbatim}
(%i1) load(nint);
(%o1) "c:/work2/nint.mac"
(%i2) ee : integrate(exp(x^5),x,1,2);
(%o2) ((-1)^(4/5)*gamma_incomplete(1/5,-32)-(-1)^(4/5)
                                            *gamma_incomplete(1/5,-1))/5
(%i3) cfloat(ee);
(%o3) 1.0132394896940175E+12-1.9531250000000001E-4*%i
(%i4) fbfloat(ee,32);
(%o4) 1.0512336862202133E-20*%i+1.0132394896940183E+12
(%i5) fchop(%);
(%o5) 1.0132394896940183E+12
(%i6) nint(exp(x^5),x,1,2);
(%o6) 1.0132394896940183E+12
(%i7) noutL;
(%o7) [integrate,1.0132394896940183E+12]
(%i8) quad(exp(x^5),x,1,2);
(%o8) 1.0132394896940144E+12
(%i9) noutL;
(%o9) [qag,1.0132394896940144E+12,0.011249218166387,155,0]
\end{myVerbatim}
In the above output, the global parameter \mv|noutL| includes the method and the
 result(s) returned by the method.\\
 
\noindent The global parameter \mv|nargL| is a list which includes both the
method and the arguments used by the method.
 The names of the integration variables actually used are
automatically generated by the code to allow global assumptions
to remain in force after the computation, hence the weird names.
\begin{myVerbatim}
(%i10) nint(exp(x^5),x,1,2);
(%o10) 1.0132394896940183E+12
(%i11) nargL;
(%o11) [integrate,%e^wx62375^5,wx62375,1,2]
(%i12) quad(exp(x^5),x,1,2);
(%o12) 1.0132394896940144E+12
(%i13) nargL;
(%o13) [qag,%e^x76014^5,x76014,1.0,2.0,3,limit = 800]
\end{myVerbatim}
\subsubsection{Turning Off Automatic Chop of Answer}
The global parameter \mv|dochop| is set to \mv|true| by the package and
 causes automatic chopping of the answer. Set \mv|dochop| to false to
 prevent automatic chopping of the answer.
\begin{myVerbatim}
(%i1) load(nint);
(%o1) "c:/work2/nint.mac"
(%i2) dochop;
(%o2) true
(%i3) nint(exp(x^5),x,1,2);
(%o3) 1.0132394896940183E+12
(%i4) dochop:false$
(%i5) nint(exp(x^5),x,1,2);
(%o5) 1.0512336862202133E-20*%i+1.0132394896940183E+12
(%i6) dochop:true$
(%i7) nint(exp(x^5),x,1,2);
(%o7) 1.0132394896940183E+12
\end{myVerbatim}
\subsubsection{Changing the Chop Value Used by \textbf{nint}}
The global parameter \mv|_small%| holds the small chop value; numbers whose absolute
  values are smaller than \mv|_small%| are automatically ``chopped'' by \mv|fchop|.
 Set \mv|_small%| to smaller values if you want less chopping.
\begin{myVerbatim}
(%i8) method : false$
(%i9) nint(exp(x^5),x,1,2);
(%o9) 1.0132394896940183E+12
(%i10) _small%;
(%o10) 1.0E-14
(%i11) _small% : 1.0e-30$
(%i12) nint(exp(x^5),x,1,2);
(%o12) 1.0512336862202133E-20*%i+1.0132394896940183E+12
\end{myVerbatim}
\newpage
\subsection{Arbitrary Precision One Dimensional Quadrature with \textbf{apnint} and \textbf{apquad}}
\subsubsection*{Syntax}
The new Ch. 8 functions \mv|apnint| (arbitrary precision numerical integration which first tries
   integrate)  and \mv|apquad| (arbitrary precision quadrature which avoids integrate) are defined
   in the file \mv|apnint.mac| (which loads \mv|tsquad.mac| and \mv|dequad.mac|).
To use \mv|apnint| and/or \mv|apquad|, you should first load the\\
 \mv|nint.mac| package, and then the \mv|apnint.mac| package.\\

\noindent Both functions have the same syntax:
\begin{myVerbatim2}
     apnint (expr, x, x1,x2,rp,wp)
     apquad (expr, x, x1,x2,rp,wp)
\end{myVerbatim2}
  in which \mv|expr| depends on the variable \mv|x| (of course you can use any
  symbol), \mv|x1| is less than \mv|x2|, \mv|x1| must evaluate to a finite real number, but
  \mv|x2| can either be finite (evaluating to a real number) or the symbol \mv|inf|.\\
  
\noindent \mv|rp| is the ``requested precision'' (``precision goal'', the number of
   trustworthy digits wanted in the result), and
  \mv|wp| is the ``working precision'' (\mv|fpprec| is set to \mv|wp| during the
  course of the calculation).\\
  
\noindent These two functions do not set values of \mv|noutL| and \mv|nargL|, but
 you can set \mv|method| to \mv|true| to have a short printout of the method used.
 The finite numerical case uses a new function \mv|tsquad| (which accepts a complex
  integrand), defined in the file \mv|tsquad.mac|.\\
  
\noindent The non-finite numerical case uses a new function \mv|dequad| (which also
  accepts a complex integrand), defined in the file \mv|dequad.mac|.\\
  
\noindent Both of these new functions return the result as a bigfloat number.
In general, \mv|bfloat(expr)| returns a bigfloat number whose precision depends on the
 current value of \mv|fpprec|. Both package functions \mv|apnint| and \mv|apquad| worry
about setting the value of \mv|fpprec| locally to the values needed corresponding
to the \mv|wp| arg. You do not have to set the global value of \mv|fpprec| yourself,
although that global value will not affect the precision of the calculation: only
the supplied value \mv|wp| affects the precision of the actual calculation.\\

\noindent Here are some examples of bigfloat number returns 
  (the default value of \mv|fpprec| is \mv|16|).
When comparing two bigfloat numbers, you should set the
  value of \mv|fpprec| to a large enough value.
Using \mv|float| to convert a bigfloat number to an ordinary
  floating point number leads to a loss of significant digits.
(See Ch. 9 for more discussion of bigfloats.)
\begin{myVerbatim}
(%i1) fpprec;
(%o1) 16
(%i2) pi16 : bfloat(%pi);
(%o2) 3.141592653589793b0
(%i3) pi30 : bfloat(%pi),fpprec:30;
(%o3) 3.14159265358979323846264338328b0
(%i4) fpprec;
(%o4) 16
(%i5) abs(pi16 - pi30);
(%o5) 0.0b0
(%i6) abs(pi16 - pi30),fpprec:40;
(%o6) 1.144237745221949411169021904042519539695b-17
(%i7) pi30f : float(pi30);
(%o7) 3.141592653589793
\end{myVerbatim}
\newpage
\begin{myVerbatim}
(%i8) abs(pi16 - pi30f),fpprec:40;
(%o8) 1.1102230246251565404236316680908203125b-16
\end{myVerbatim}
\subsubsection*{An Accuracy Test}
A test used in Ch.9 is the known integral
\begin{equation}
\mathbf{\int_{0}^{1} \sqrt{t}\,\boldsymbol{\ln}(t)\,dt = -4/9}
\end{equation}
You need to load both the \mv|nint.mac| package, and then the
 \mv|apnint| package (the latter loads \mv|tsquad.mac| and
 \mv|dequad.mac|).
\begin{myVerbatim}
(%i1) (load(nint),load(apnint));
   _kmax% =  8   _epsfac% =  2 
(%o1) "c:/work2/apnint.mac"
(%i2) integrate(sqrt(x)*log(x),x,0,1);
(%o2) -4/9
(%i3) tval : bfloat(%),fpprec:45;
(%o3) -4.44444444444444444444444444444444444444444445b-1
(%i4) apval : apquad(sqrt(x)*log(x),x,0,1,30,40);
 construct _yw%[kk,fpprec] array for kk = 
  8  and fpprec =  40   ...working... 
(%o4) -4.444444444444444444444444444444444444445b-1
(%i5) abs(apval - tval),fpprec:45;
(%o5) 4.46422161518439410412204461320543256312937649b-41
\end{myVerbatim}
Requesting thirty digit accuracy with forty digit arithmetic returns a value
  for this integral which has about forty digit precision.
See Ch.9 for other tests using known integrals.\\

\noindent The loading of the \mv|apnint.mac| file causes the loading of other
  files of the package. The loading of the package file \mv|tsquad.mac| 
  produces the printout: \mv|_kmax% =  8   _epsfac% =  2| seen in the above
  example, which warns that two global parameters are set by the package.
  Ch.9 discusses the meaning of those parameters, and the possibility of
  changing them to fit your special problem.\\
  
\noindent An array of transformation coefficients is constructed, using
  40 digit arithmetic, in the above example. Once that array has been
  constructed, other numerical integrals can be evaluated using the same
  set of coefficients (as long as the same working precision \mv|wp| is
  requested).\\
  
\noindent If you change the value of \mv|wp|, a new array of coefficients is
constructed, which is available for use with other integrals (the original
  40 digit array is still available also).
\begin{myVerbatim}
(%i6) apquad(sqrt(x)*log(x),x,0,1,20,30);
 construct _yw%[kk,fpprec] array for kk = 
  8  and fpprec =  30   ...working... 
(%o6) -4.44444444444444444444444444445b-1
(%i7) apquad(sqrt(x)*log(x),x,0,1,30,40);
(%o7) -4.444444444444444444444444444444444444445b-1
\end{myVerbatim}
\newpage
\subsubsection*{Some Examples}
Using \mv|apquad| forces use of a numerical method discussed in Ch.9,
  instead of first trying \mv|integrate|.
The \mv|expr| to be integrated can be complex.
\begin{myVerbatim}
(%i8) apquad(sin(x)*exp(%i*x),x,0,2,20,30);
(%o8) 1.18920062382698206284315977363b0*%i+4.13410905215902978659792045774b-1
\end{myVerbatim}
However this integral can be done using \mv|integrate|, and Maxima
  is successful in obtaining a \mv|bfloat| value of the symbolic result:
\begin{myVerbatim}
(%i9) method:true$
(%i10) apnint(sin(x)*exp(%i*x),x,0,2,20,30);
 integrate 
(%o10) 1.18920062382698206284315977363b0*%i+4.13410905215902978659792045774b-1
\end{myVerbatim}
Here is an example of a non-finite integral with a complex integrand.
We force use of a numerical method by using \mv|apquad|:
\begin{myVerbatim}
(%i11) apquad(exp(-x +%i*x),x,0,inf,20,30);
dequad 
(%o11) 5.0b-1*%i+5.0b-1
\end{myVerbatim}
The package function \mv|dequad| is defined in the file \mv|dequad.mac|,
 loaded by \mv|apnint.mac|, and can handle a complex integrand, passing the
  real and imaginary parts separately to the Ch.9 function \mv|quad_de|
  (the latter assumes a real integrand).\\
  
\noindent As in the previous example, using \mv|apnint| produces an answer from \mv|integrate|:
\begin{myVerbatim}
(%i12) apnint(exp(-x +%i*x),x,0,inf,20,30);
 integrate 
(%o12) 5.0b-1*%i+5.0b-1
\end{myVerbatim}
Here is an example of an integrand containing a Bessel function, and \mv|integrate|
  cannot find the definite integral over the non-finite domain, nor can \mv|integrate|
  find the indefinite integral:
\begin{myVerbatim}
(%i13) integrate(bessel_j(1,x)*exp(-x),x,0,inf);
(%o13) 'integrate(bessel_j(1,x)*%e^-x,x,0,inf)
(%i14) integrate(bessel_j(1,x)*exp(-x),x);
(%o14) 'integrate(bessel_j(1,x)*%e^-x,x)
\end{myVerbatim}
If we force the use of the numerical bigfloat methods, the integrand is first tested
  to see if evaluation at a bigfloat specified point returns a bigfloat number, and
  for this example the Bessel function factor is not reduced:
\begin{myVerbatim}
(%i15) bfloat(bessel_j(1,0.5b0)*exp(-0.5b0));
(%o15) 6.065306597126334b-1*bessel_j(1.0b0,5.0b-1)
\end{myVerbatim}
  and \mv|apquad| returns \mv|false|.
\begin{myVerbatim}
(%i16) apquad(bessel_j(1,x)*exp(-x),x,0,inf,20,30);
apquad: cannot obtain bfloat value from integrand 
(%o16) false
\end{myVerbatim}
The presence of \mv|sqrt| and/or \mv|log| in an integrand causes \mv|apnint| to bypass \mv|integrate|,
(just as happens with \mv|nint|).
\begin{myVerbatim}
(%i17) apnint(log(1/x)/sqrt(%i*x),x,0,1,20,30);
tsquad 
(%o17) 2.82842712474619009760337744842b0-2.82842712474619009760337744842b0*%i
\end{myVerbatim}
\newpage
Here is an example of a complex integrand in which \mv|quad| can get a numerical
  answer (about 15 digit precision), but \mv|apquad| can only get an answer for the
  \mv|realpart| of the integrand:
\begin{myVerbatim}
(%i18) quad(log(-3+%i*x),x,-2,3);
   quad_qag 
   quad_qags 
(%o18) 2.449536971144524*%i+6.02070929514083
(%i19) quad(realpart(log(-3+%i*x)),x,-2,3);
   quad_qag 
(%o19) 6.02070929514083
(%i20) quad(imagpart(log(-3+%i*x)),x,-2,3);
   quad_qags 
(%o20) 2.449536971144524
(%i21) apnint(realpart(log(-3+%i*x)),x,-2,3,20,30);
tsquad 
(%o21) 6.02070929514083135694888711387b0
(%i22) apnint(imagpart(log(-3+%i*x)),x,-2,3,20,30);
quad_ts: vdiffnew > vdiffold before vdiff < eps0 reached 
quad_ts: abort calc. 
(%o22) false
(%i23) imagpart(log(-3+%i*x));
(%o23) atan2(x,-3)
\end{myVerbatim}
The bigfloat calculation for the imaginary part of the integrand was aborted when
  the differences between the current estimate and the previous estimate of the integral
  started growing (instead of decreasing steadily).
\subsection{Using the Wolfram Alpha Web Site for Integrals}
The Wolfram Alpha web page, \mv|http://www.wolframalpha.com|,
 allows free one-line commands (integrals, derivatives, plots, etc.,)
 which may require translating Maxima syntax into Mathematica syntax.\\
 
\noindent A web page with some translation from Maxima to Mathematica is
\begin{myVerbatim2}
  http://www.math.harvard.edu/computing/maxima/
\end{myVerbatim2}
A larger comparison of Maxima, Maple, and Mathematica syntax is at
\begin{myVerbatim2}
http://beige.ucs.indiana.edu/P573/node35.html
\end{myVerbatim2}
Maxima syntax for 1d symbolic integration is
\begin{myVerbatim}
(%i13) integrate(cos(x),x,0,1);
(%o13) sin(1)
(%i14) float(%);
(%o14) 0.8414709848079
\end{myVerbatim}
\noindent The corresponding Mathematica request would be
(note the curly brackets):
\begin{myVerbatim2}
Integrate[Cos[x], {x, 0, 1}]
\end{myVerbatim2}
 for a symbolic answer, and
 \begin{myVerbatim2}
 NIntegrate[Cos[x], {x, 0, 1}]
 \end{myVerbatim2}
produces a numerical result corresponding to Maxima's
\begin{myVerbatim2}
 float(integrate(cos(x),x,0,1))
\end{myVerbatim2}
Mathematica also has the function \mv|N| which computes the
  numerical value of \mv|expr| with the syntax:
\begin{myVerbatim2}
 N[ expr ]
\end{myVerbatim2}
or
\begin{myVerbatim2}
  expr //N
\end{myVerbatim2}
Two examples of numerical two dimensional integration in Mathematica
 syntax are:
\begin{myVerbatim2}
  NIntegrate[1/Sqrt[x + y], {x,0,1},{y,0,1} ]
  NIntegrate[Sin[ x*y ],{x,0,1}, {y,0,1} ]
\end{myVerbatim2}
\subsection{Direct Quadpack Use of \textbf{quad\_qags}, \textbf{quad\_qag}, and \textbf{quad\_qagi}}
The quadpack subroutine package originally was a set of Fortran 
   subroutines, described in the book:
\begin{myVerbatim2}
Quadpack: A Subroutine Package for Automatic Integration,
by R. Piessens, E. de Doncker-Kapenga, C.W. Uberhuber,
     and D.K. Kahaner,
Springer-Verlag, 1983
\end{myVerbatim2}
and designed for the computation of 1d integrals.
The Maxima developer Raymond Toy has adapted the package to Maxima via a
  translation from Fortran to Common Lisp (using f2cl) combined with a
  special interface for the Maxima user.\\
  
\noindent See the Maxima Help Manual for an extensive discussion in Sec. 19.3
 (Sec. 19 is Integration).
\subsubsection{Syntax for Quadpack Functions}  
The integrand \mv|expr| supplied as the first argument to a quadpack function must evaluate
  to a real number throughout the requested range of the independent variable.
All the quadpack functions (``q'' is for ``quadrature''), except two, are called using the syntax:
\begin{myVerbatim2}
   quad_qaxx ( expr, var, a, b, (other-required-args), optional-args)
\end{myVerbatim2}
The exceptions are the Cauchy principle value integration routine \mv+quad_qawc+
  which does the integral\\
  $\mathbf{\int_{a}^{b} f(x)/(x - c) \, dx}$, 
  using the syntax \mvs+quad_qawc(expr,var,c,a,b,optional-args)+,\\
  and the cosine or sine Fourier transform 
  integration routine \mv+quad_qawf+
  which does integrals\\
  $\mathbf{\int_{a}^{\infty} f(x)\,\boldsymbol{\cos}(\boldsymbol{\omega}\,x)\,dx}$ \, or \,
  $\mathbf{\int_{a}^{\infty} f(x)\,\boldsymbol{\sin}(\boldsymbol{\omega}\,x)\,dx}$ 
  using the syntax\\ 
    \mvs+   quad_qawf(expr,var,a,omega,trig,optional-args)+.
\subsubsection{Ouput List of Quadpack Functions and Error Code Values}
All Quadpack functions return a \textbf{list} of four elements: 
\begin{myVerbatim2}
 [num-val, est-abs-error, number-integrand-evaluations, error-code].
\end{myVerbatim2}
%\newpage 
The \textbf{error-code} (the fourth element of the returned list) can
    have the values: 	
\begin{itemize}
\item[] 0 - no problems were encountered
\item[] 1 - too many sub-intervals were done
\item[] 2 - excessive roundoff error is detected
\item[] 3 - extremely bad integrand behavior occurs
\item[] 4 - failed to converge
\item[] 5 - integral is probably divergent or slowly convergent
\item[] 6 - the input is invalid
\end{itemize}  
%\newpage
\subsubsection{Integration Rule Parameters and Optional Arguments}  
All the Quadpack functions (except one) include   \textbf{epsrel}, \textbf{epsabs}, 
  and \textbf{limit} as possible optional types of allowed arguments to override
  the default values of these parameters: \textbf{epsrel = 1e-8, epsabs = 0.0, limit = 200}.\\  
To override the \textbf{epsrel} default value, for example, you would add the
  optional argument \textbf{epsrel = 1e-6} or \textbf{epsrel=1d-6} (both have the
  same effect).\\

These Quadpack functions apply an integration rule adaptively until an estimate 
   of the integral of \textbf{expr} over the interval \textbf{(a, b)} is achieved within
   the desired absolute and relative error limits, \textbf{epsabs} and \textbf{epsrel}.
With the default values \textbf{epsrel = 1e-8, epsabs = 0.0}, only \textbf{epsrel} plays
  a role in determining the convergence of the integration rules used, and this
  corresponds to getting the estimated relative error of the returned answer
  smaller than \textbf{epsrel}.\\
  
If you override the defaults with the two optional arguments (in any order)
  \textbf{epsrel = 0.0, epsabs = 1e-8}, for example, the value of \textbf{epsrel} will be
  ignored and convergence will have been achieved if the estimated absolute error
  is less than \textbf{epsabs}.\\
  
%\small
The integration region is divided into subintervals, and on each iteration the
   subinterval with the largest estimated error is bisected. 
This ``adaptive method'' reduces the overall error rapidly, as the subintervals become concentrated around
   local difficulties in the integrand.\\
   
The Quadpack parameter \textbf{limit}, whose default value is \textbf{200}, is the maximum
  number of subintervals to be used in seeking a convergent answer. To increase that
  limit, you would insert \textbf{limit=300}, for example.
\newpage
\subsubsection{\textbf{quad\_qags} for a Finite Interval}  \label{qags}
The ``s'' on the end of \textbf{qags} is a signal that \mv+quad_qags+ 
  has extra abilities in dealing with functions which have integrable 
  \textbf{singularities}, but even if your function has no singular behavior,
  \mv+quad_qags+ is still the best choice due to the sophistication
  of the quadrature algorithm used. Use the syntax
\begin{myVerbatim2}
         quad_qags ( expr, var , a, b, [ epsrel, epsabs, limit ] ) 
\end{myVerbatim2}
%\newpage  
where the keywords inside the brackets indicate possible optional
  arguments (entered in any order), such as \textbf{epsrel = 1e-6}.
  
Thus the approximate numerical value of $\mathbf{\int_{0}^{1} e^{x^{2}} \, dx}$ would be the first
  element of the list returned by \mvs|quad_qags (exp (x^2), x, 0, 1)|.
\begin{myVerbatim}
(%i1) display2d:false;
(%o1) false
(%i2) domain;
(%o2) real
(%i3) quad_qags(exp (x^2),x,0,1);
(%o3) [1.462651745907182,1.6238696453143376E-14,21,0]
\end{myVerbatim}    
If you call \mv|quad_qags| with an unbound parameter in the integrand,
  a noun form will be returned which will tell you all the defaults being used.
\begin{myVerbatim}
(%i4) quad_qags(a*x,x,0,1);
(%o4) quad_qags(a*x,x,0,1,epsrel = 1.0E-8,epsabs = 0.0,limit = 200)
\end{myVerbatim}
There is a more efficient Quadpack function available, \mv|quad_qaws|, for integrals
  which have end point algebraic and/or logarithmic singularites associated with 
  of the form $\mathbf{\int_{a}^{b} f(x)\, w(x)\,dx}$, if the ``weight function'' $\mathbf{w(x)}$ has
  the form (note: $\mathbf{f(x)}$ is assumed to be well behaved)
\begin{equation}
\mathbf{w(x) = (x-a)^{\boldsymbol{\alpha}}\,(b-x)^{\boldsymbol{\beta}} \,\boldsymbol{\ln}(x-a)\, \boldsymbol{\ln}(b-x)}
\end{equation}
where $\boldsymbol{\alpha} > -1$ and $\boldsymbol{\beta} > -1$ for convergence.
See Sec \ref{qaws} for more information on \mv|quad_qaws| and its use.    
% \newpage
\subsubsection*{Example 1}  
Here is an example of the use of \mv+quad_qags+ to compute the numerical
  value of the integral $\mathbf{\int_{0}^{1} \sqrt{x}\,\boldsymbol{\ln} (1/x) \, dx}$.
The integrand $\mathbf{g = x^{1/2}\,\boldsymbol{\ln}(1/x) = - x^{1/2}\,\boldsymbol{\ln}(x)}$ has the limiting
  value $\mathbf{0}$ as $\mathbf{x \rightarrow 0}$ from the positive side.
Thus the integrand is not singular at $\mathbf{x=0}$, but it is rapidly changing near $\mathbf{x=0}$
  so an efficient algorithm is needed.
(Our Example 2 will work with an integrand which is singular at $\mathbf{x=0}$.)\\
  
\noindent Let's check the limit at $\mathbf{x=0}$ and plot the function.
\begin{myVerbatim}
(%i5) g:sqrt(x)*log(1/x)$
(%i6) limit(g,x,0,plus);
(%o6) 0
(%i7) (load(draw),load(qdraw))$
(%i8) qdraw( ex(g,x,0,1) )$
\end{myVerbatim}
\newpage
\noindent Here is that plot.
\smallskip
\begin{figure} [h]  
   \centerline{\includegraphics[scale=.6]{ch7p2.eps} }
	\caption{ $\mathbf{x^{1/2}\,\boldsymbol{\ln}(1/x)}$  }
\end{figure} 
% eps code
% (%i49) qdraw( ex(x^(1/2)*log(1/x),x,0,1),
%          pic(eps,"ch7p2",font("Times-Bold",20) ) )$

\noindent Now we try out \mv|quad_qags| on this integral and compare with the
  exact answer returned by \mv|integrate|, converted to 20 digit accuracy
  using \mv|bfloat|.
\begin{myVerbatim}
(%i9) fpprintprec:8$
(%i10) tval : block([fpprec:20], bfloat(integrate(g,x,0,1)) );
(%o10) 4.4444444b-1
(%i11) qlist:quad_qags(g,x,0,1 );
(%o11) [0.444444,4.93432455E-16,315,0]
(%i12) block ([fpprec:20], bfloat(abs(first(%) - tval)));
(%o12) 8.0182679b-17
\end{myVerbatim} 
The first element of the returned list \textbf{qlist} is the approximate numerical value of the
  integral.
We have used \textbf{integrate} together with \textbf{bfloat} and \textbf{fpprec} 
 to generate the ``true value'' good to about \textbf{20} digits (see Chapter 9),
  and we see that the absolute error of the answer returned by \mv+quad_qags+ is
  about $\mathbf{10^{-16}}$.
The second element of \textbf{qlist} is the \textbf{estimated} absolute error of the returned answer.
The third element shows that  \textbf{315} integrand evaluations were needed
  to attain the requested (default) relative error \textbf{epsrel = 1e-8}.
The fourth element is the error code value \textbf{0} which indicates no problems were
  found.\\

\noindent The algorithm used to find an approximation for this integral does not know the
  ``exact'' (or true) answer, but does have (at each stage) an estimate of the answer
  and an estimate of the likely error of this estimated answer (gotten by comparing
  the new answer with the old answer, for example), and so can compute an estimated
  relative error (est\_abs\_err/est\_answer) which the code checks against the requested
  relative error goal supplied by the parameter \textbf{epsrel}. \\
  
\noindent We are assuming that the defaults are being used, and hence \textbf{epsabs} has the
  value \textbf{0.0}, so that the convergence criterion used by the code is 
\begin{equation}
\mathbf{est\_rel\_err} \leq \mathbf{epsrel}
\end{equation}
or
\begin{equation}
\frac{\mathbf{est\_abs\_err}}{\mathbf{est\_answer}} \leq \mathbf{epsrel}
\end{equation}
or
\begin{equation}
\mathbf{est\_abs\_err} \leq  \mathbf{epsrel} \times \mathbf{est\_answer}
\end{equation}
\newpage
\noindent We can check that the values returned by \mv+quad_qags+ are at least
  consistent with the advertised convergence criterion. 
\begin{myVerbatim}
(%i13) est_answer : first(qlist);
(%o13)                              0.444444
(%i14) est_abs_err : second(qlist);
(%o14)                           4.93432455E-16
(%i15) est_rel_err : est_abs_err/est_answer;
(%o15)                           1.11022302E-15
\end{myVerbatim} 
We see that the \mv+est_rel_err+ is much less than \textbf{epsrel}.
\subsubsection*{Example 2}    
Here we use \mv+quad_qags+ with an integrand which is singular at $\mathbf{x=0}$, 
  but the singularity is an ``integrable singularity''.
We want to calculate $\mathbf{\int_{0}^{1} \boldsymbol{\ln}(\boldsymbol{\sin}(x)) \, dx}$.
\textbf{integrate} returns an
  ``exact'' answer involving the dilogarithm
  function with a complex argument (see below).\\
  
\noindent We first show that the integrand is singular at $\mathbf{x=0}$, first using
  Maxima's \textbf{limit} function and then using \textbf{taylor}.
We then use the approximate integrand for small $\mathbf{x}$ to show that the
  indefinite integral is a function of $\mathbf{x}$ which has a finite limit at
  $\mathbf{x=0}$. 
\begin{myVerbatim}
(%i1) display2d:false$
(%i2) limit(log(sin(x)),x,0,plus);
(%o2) minf
(%i3) ix : taylor(log(sin(x)),x,0,2);
(%o3) +log(x)-x^2/6
(%i4) int_ix:integrate(ix,x);
(%o4) x*log(x)-x^3/18-x
(%i5) limit(int_ix,x,0,plus);
(%o5) 0
(%i6) assume(eps>0,eps<1)$
(%i7) integrate(ix,x,0,eps);
(%o7) (18*eps*log(eps)-eps^3-18*eps)/18
(%i8) expand(%);
(%o8) eps*log(eps)-eps^3/18-eps
(%i9) limit(eps*log(eps),eps,0,plus);
(%o9) 0
\end{myVerbatim} 
Output \textbf{\%o2} indicates that the integrand $\mathbf{\rightarrow -\infty}$ 
  as $\mathbf{x \rightarrow 0^{+}}$.
To see if this is an ``integrable singularity'', we examine the integrand for $\mathbf{x}$ positive but
  close to $\mathbf{0}$, using a Taylor series expansion.
We let \textbf{eps} represent $\boldsymbol{\epsilon}$, a small positive number to be
  used as the upper limit of a small integration interval.\\
  
\noindent The approximate integrand, \textbf{ix}, shows a logarithmic singularity at $\mathbf{x=0}$.
The indefinite integral of the approximate integrand, \textbf{int\_ix}, is finite as
  $\mathbf{x \rightarrow 0^{+}}$.
And finally, the integral of the approximate integrand over $\boldsymbol{(0,\epsilon)}$ is finite.
Hence we are dealing with an integrable singularity, and we use \mv+quad_qags+
  to evaluate the numerical value of the exact integrand over $\mathbf{[0,1]}$. 
\begin{myVerbatim}
(%i10) quad_qags(log(sin(x)),x,0,1);
(%o10) [-1.056720205991585,1.1731951032784962E-15,231,0]
\end{myVerbatim}
\newpage
\noindent Use of \textbf{integrate}  with this integrand-domain returns an expression involving
the dilogarithm function\\
 \mv|li[2](z)| in which \mv|z| is a complex number.
At present, Maxima cannot find numerical values of the dilogarithm function
  for non-real arguments.
\begin{myVerbatim}
(%i11) float(li[2](1));
(%o11) 1.644934066848226
(%i12) float(li[2](1+%i));
(%o12) li[2](%i+1.0)
(%i13) integrate(log(sin(x)),x,0,1);
(%o13) -%i*atan(sin(1)/(cos(1)+1))-%i*atan(sin(1)/(cos(1)-1))+log(sin(1))
                                  -log(2*cos(1)+2)/2-log(2-2*cos(1))/2
                                  +%i*li[2](%e^%i)+%i*li[2](-%e^%i)
                                  -%i*%pi^2/12+%i/2
\end{myVerbatim}
\subsubsection{\textbf{quad\_qags} for Double Integrals}
In Sec.7.6 of Ch.7 we presented the notation (for an exact symbolic double integral):\\

\noindent To evaluate the exact symbolic double integral
\begin{equation}
\int_{u1}^{u2} du \, \int_{v1(u)}^{v2(u)} dv\,f(u,v)  \equiv 
  \int_{u1}^{u2} \left( \int_{v1(u)}^{v2(u)} f(u,v) \, dv \right) \, du
\end{equation}
we use \textbf{integrate} with the syntax: 
\begin{myVerbatim2}
       integrate( integrate( f(u,v), v, v1(u), v2(u) ), u, u1, u2 )
\end{myVerbatim2} 
in which \textbf{f(u,v)} can either be an expression depending on the
  variables \textbf{u} and \textbf{v}, or  a Maxima function, and
  likewise \textbf{v1(u)} and \textbf{v2(u)} can either be expressions depending
  on \textbf{u} or Maxima functions.\\
  
\noindent Both \textbf{u} and \textbf{v} are ``dummy variables'', since the value of the
  resulting double integral does not depend on our choice of symbols for
  the integration variables; we could just as well use \textbf{x} and \textbf{y}.\\
  
\noindent To use \textbf{quad\_qags} we use
\begin{myVerbatim2}
   quad_qags ( quad_qags ( f(u,v), v, v1(u), v2(u) )[1], u, u1, u2 )
\end{myVerbatim2}
\subsubsection*{Example 1}
For our first example, we use \textbf{quad\_qags} on the double integral:
\begin{equation}
\mathbf{\int_{1}^{3} \left( \int_{0}^{x/2} \frac{x\,y}{x+y} \, dy \right) dx }
\end{equation}
comparing with the \textbf{integrate} result,
  first using the absolute error, then the relative error. 
\begin{myVerbatim}
(%i1) (fpprintprec:8,display2d:false)$
(%i2) g : x*y/(x+y)$
(%i3) tval : (block [fpprec:20],bfloat (integrate (integrate (g,y,0,x/2),x,1,3) ));
Is  x  positive, negative, or zero?
p;
(%o3) 8.1930239b-1
\end{myVerbatim}
\newpage
\begin{myVerbatim}
(%i4) quad_qags ( quad_qags (g,y,0,x/2)[1],x,1,3);
(%o4) [0.819302,9.09608385E-15,21,0]
(%i5) block([fpprec:20], bfloat( abs (%[1] - tval)));
(%o5) 6.1962154b-17
(%i6) %/tval;
(%o6) 7.5627942b-17
\end{myVerbatim}
Bear in mind that the quadpack functions (at present) have, at most, 16 digit accuracy.
For a difficult numerical integral, the accuracy can be much less than that.
\subsubsection*{Example 2}
For our second example, we consider the double integral
\begin{equation}
\mathbf{\int_{0}^{1} \left( \int_{1}^{2+x} e^{x - y^{2}}\, dy \right) \, dx }
\end{equation}  
\begin{myVerbatim}
(%i7) g : exp(x-y^2)$
(%i8) tval : block([fpprec:20],bfloat(integrate( integrate(g,y,1,2+x),x,0,1)));
(%o8) 2.3846836b-1
(%i9) quad_qags( quad_qags(g,y,1,2+x)[1],x,0,1);
(%o9) [0.238468,2.64753066E-15,21,0]
(%i10) block([fpprec:20],bfloat(abs (%[1] - tval)));
(%o10) 4.229659b-18
(%i11) %/tval;
(%o11) 1.7736772b-17
\end{myVerbatim} 
The Maxima coordinator, Robert Dodier has warned about the 
 lack of accuracy diagnostics with the inner integral done 
 by \textbf{quad\_qags}:
\begin{quote}
A nested numerical integral like this has a couple of drawbacks,
whatever the method. (1) The estimated error in the inner integral
isn't taken into account in the error estimate for the outer.
(2) Methods specifically devised for multi-dimensional integrals
are typically more efficient than repeated 1-d integrals.
\end{quote}
%\newpage
\subsubsection{ \textbf{quad\_qag} for a General Oscillatory Integrand}
The function \mv+quad_qag+ is useful primarily due to its
  ability to deal with functions with some general oscillatory behavior.\\

\noindent The ``key'' feature to watch out for is the required fifth slot argument which
  the manual calls ``key''.
This slot can have any integral value between $1$ and $6$ inclusive, with the
  higher values corresponding to ``higer order Gauss-Kronrod'' integration rules
  for more complicated oscillatory behavior.\\

\noindent  Use the syntax:
\begin{myVerbatim2}
        quad_qag( expr, var, a, b, key, [epsrel, epsabs, limit] )
\end{myVerbatim2}
Thus the approximate numerical value of $\mathbf{\int_{0}^{1} e^{x^{2}} \, dx}$ would be the first
  element of the list returned by\\
     \mvs+quad_qag ( exp ( x^2 ), x, 0, 1, 3)+  using the ``key'' value \textbf{3}.
%\newpage
\subsubsection*{Example 1}
Since the main feature of \mv+quad_qag+ is the ability to select a high order
  quadrature method for a generally oscillating integrand, we begin by comparing
  \mv+quad_qag+ with \mv+quad_qags+ for such a case.
We consider the integral
 $\mathbf{\int_{0}^{1} \boldsymbol{\cos}(50\,x) \, \boldsymbol{\sin}(3\,x) \, e^{-x} \, dx}$. 
\begin{myVerbatim}
(%i1) (fpprintprec:8,display2d:false)$
(%i2) f : cos(50*x)*sin(3*x)*exp(-x)$
(%i3) tval : block ([fpprec:20],bfloat( integrate (f,x,0,1) ));
(%o3) -1.9145466b-3
(%i4) (load(draw),load(qdraw))$
(%i5) qdraw( ex(f,x,0,1) )$
\end{myVerbatim} 
Here is that plot:
\smallskip
\begin{figure} [h]
   \centerline{\includegraphics[scale=.6]{ch7p20.eps} }
	\caption{ $\mathbf{\boldsymbol{\cos}(50\,x)\,\boldsymbol{\sin}(3\,x)\,e^{-x}}$  }
\end{figure} 
%\newpage
\noindent and here we make the comparison. 
\begin{myVerbatim}
(%i6) quad_qag(f,x,0,1,6);
(%o6) [-0.00191455,2.86107558E-15,61,0]
(%i7) block ([fpprec:20], bfloat(abs(first(%) - tval)));
(%o7) 9.8573614b-17
(%i8) %/tval;
(%o8) -5.148666b-14
(%i9) quad_qags(f,x,0,1 );
(%o9) [-0.00191455,2.8536E-15,315,0]
(%i10) block ([fpprec:20], bfloat(abs(first(%) - tval)));
(%o10) 1.1804997b-16
(%i11) %/tval;
(%o11) -6.1659493b-14
\end{myVerbatim} 
We see that both methods returned results with about the same relative error, but that
  \mv+quad_qag+ needed only about one fifth the number of integrand
  evaluations as compared with \mv+quad_qags+.
This extra efficiency of \mv+quad_qag+ for this type of integrand may be
  of interest in certain kinds of intensive numerical work.
Naturally, the number of integrand evaluations needed does not necessarily translate simply
  into time saved, so timing trials would be appropriate when considering this option.
\newpage
\subsubsection*{Example 2}
We compare \mv+quad_qag+ and \mv+quad_qags+ with the numerical evaluation
  of the integral $\mathbf{\int_{0}^{1} e^{x} \, dx}$, which has neither special oscillatory
  behavior (which \mv+quad_qag+ might help with) nor singular behavior 
  (which \mv+quad_qags+ might help with). 
\begin{myVerbatim}
(%i12) tval : block ([fpprec:20], bfloat( integrate (exp(x),x,0,1) ));
(%o12) 1.7182818b0
(%i13) quad_qag(exp(x),x,0,1,6);
(%o13) [1.7182818,1.90767605E-14,61,0]
(%i14) block ([fpprec:20], bfloat(abs(first(%) - tval)));
(%o14) 1.445648b-16
(%i15) %/tval;
(%o15) 8.413335b-17
(%i16) quad_qags(exp(x),x,0,1);
(%o16) [1.7182818,1.90767605E-14,21,0]
(%i17) block ([fpprec:20], bfloat(abs(first(%) - tval)));
(%o17) 7.7479797b-17
(%i18) %/tval;
(%o18) 4.5091437b-17
\end{myVerbatim} 
We see that using \mv+quad_qags+ for this function results in slightly smaller relative error
  while using only two thirds the number of integrand evaluations compared with \mv+quad_qag+.  
\subsubsection*{Example 3}
We compare \mv+quad_qag+ and \mv+quad_qags+ with the numerical evaluation
  of the integral $\mathbf{\int_{0}^{1} \sqrt{x} \; \boldsymbol{\ln}(x) \, dx}$, which has quasi-singular
  behavior at $\mathbf{x = 0}$. 
  See Sec.(\ref{qags}) for a plot of this integrand.
\begin{myVerbatim}
(%i19) f : sqrt(x)*log(1/x)$
(%i20) tval : block ([fpprec:20], bfloat( integrate (f,x,0,1) ));
(%o20) 4.4444444b-1
(%i21) quad_qag(f,x,0,1,3);
(%o21) [0.444444,3.17009685E-9,961,0]
(%i22) block ([fpprec:20], bfloat(abs(first(%) - tval)));
(%o22) 4.7663293b-12
(%i23) %/tval;
(%o23) 1.072424b-11
(%i24) quad_qags(f,x,0,1);
(%o24) [0.444444,4.93432455E-16,315,0]
(%i25) block ([fpprec:20], bfloat(abs(first(%) - tval)));
(%o25) 8.0182679b-17
(%i26) %/tval;
(%o26) 1.8041102b-16
\end{myVerbatim} 
We see that using \mv+quad_qags+ for this function (which has a quasi-singular behavior
  near $\mathbf{x=0}$ ) returns a much smaller relative error while using only one third the number
  of integrand evaluations compared with \mv+quad_qag+.\\
  
\noindent You can use \mv|quad_qag| for numerical double integrals, following the
  pattern we used with \mv|qaud_qags|, and you can use different methods for each of
  the two axes.
\newpage
\subsubsection{\textbf{quad\_qagi} for a Non-finite Interval  }
\tcbr{See Sec.(\ref{inf}) for a decision tree for quadrature over
  a non-finite interval}.\\

\noindent The syntax is
\begin{myVerbatim2}
    quad_qagi ( expr, var, a, b, [epsrel, epsabs, limit] ), where
            (a,b) are the limits of integration.
  Thus  you will have (omitting the optional args):
  quad_qagi (expr, var, minf, b ) with b finite,
  quad_qagi ( expr, var, a, inf ), with a finite, or
  quad_qagi (expr, var, minf, inf ).
\end{myVerbatim2}
If at least one of \textbf{(a,b)} are not equal to \textbf{(minf,inf)}, a noun
  form will be returned.\\
  
\noindent The Maxima function \mv+quad_qagi+ returns the same type of information (approx-integral,
  est-abs-error, nfe, error-code) in a list that \mv+quad_qags+ returns, and the possible
  error codes returned have the same meaning.\\  

\noindent Here we test the syntax and behavior with a simple integrand,
  first over $\mathbf{[0, \infty]}$:
\begin{myVerbatim}

(%i1) (fpprintprec:8,display2d:false)$
(%i2) tval : block ([fpprec:20],bfloat( integrate (exp(-x^2),x,0,inf) ));
(%o2) 8.8622692b-1
(%i3) quad_qagi(exp(-x^2),x,0,inf);
(%o3) [0.886227,7.10131839E-9,135,0]
(%i4) block ([fpprec:20], bfloat(abs(first(%) - tval)));
(%o4) 7.2688979b-17
(%i5) %/tval;
(%o5) 8.202073b-17
\end{myVerbatim}
 and next the same simple integrand over $\mathbf{[-\infty, 0]}$: 
\begin{myVerbatim}
(%i6) quad_qagi(exp(-x^2),x,minf,0);
(%o6) [0.886227,7.10131839E-9,135,0]
(%i7) block ([fpprec:20], bfloat(abs(first(%) - tval)));
(%o7) 7.2688979b-17
\end{myVerbatim} 
and, finally over $\mathbf{[-\infty, \infty]}$:
\begin{myVerbatim}
(%i8) tval : block ([fpprec:20],bfloat(2*tval));
(%o8) 1.7724538b0
(%i9) quad_qagi(exp(-x^2),x,minf,inf);
(%o9) [1.7724539,1.42026368E-8,270,0]
(%i10) block ([fpprec:20], bfloat(abs(first(%) - tval)));
(%o10) 1.4537795b-16
\end{myVerbatim}
\newpage
\subsubsection*{Example 1}
Here is another simple example:
\begin{myVerbatim}
(%i11) g : exp(-x)*x^(5/100)$
(%i12) tval : block ([fpprec:20],bfloat( integrate(g,x,0,inf) ));
(%o12) 9.7350426b-1
(%i13) quad_qagi(g,x,0,inf);
(%o13) [0.973504,1.2270015E-9,315,0]
(%i14) block ([fpprec:20], bfloat(abs(first(%) - tval)));
(%o14) 7.9340695b-15
(%i15) %/tval;
(%o15) 8.15001b-15
\end{myVerbatim} 
We see that the use of the default mode (accepting the default \textbf{epsrel:1d-8}
  and \textbf{epsabs:0} ) has resulted in an absolute error which is close to
  the floating point limit and a relative error of the same order of magnitude
  (because the value of the integral is of order $1$). 
\subsubsection*{Example 2}
We evaluate the integral $\mathbf{\int_{0}^{\infty}e^{-x}\,\boldsymbol{\ln}(x)\,dx = -\boldsymbol{\gamma}}$, where
  $\boldsymbol{\gamma}$ is the Euler-Mascheroni constant, \textbf{0.5772156649015329}.
Maxima has this constant available as \textbf{\%gamma}, although you need to use
  either \textbf{float} or \textbf{bfloat} to get the numerical value.
\begin{myVerbatim}
(%i16) g : exp(-x)*log(x)$
(%i17) tval : block ([fpprec:20],bfloat( integrate(g,x,0,inf) ));
(%o17) -5.7721566b-1
(%i18) quad_qagi(g,x,0,inf);
(%o18) [-0.577216,5.11052578E-9,345,0]
(%i19) block ([fpprec:20], bfloat(abs(first(%) - tval)));
(%o19) 2.6595919b-15
(%i20) %/tval;
(%o20) -4.6076226b-15
\end{myVerbatim}
\subsubsection*{Example 3}
A symmetrical version of a Fourier transform pair is defined by the equations
\begin{equation}
\mathbf{g(x) = \frac{1}{\sqrt{2\, \boldsymbol{\pi}}} \,\int_{-\infty}^{\infty} G(k)\, e^{i\,k\,x}\,dk}
\end{equation}
\begin{equation}
\mathbf{G(k) = \frac{1}{\sqrt{2\,\boldsymbol{\pi}}} \, \int_{-\infty}^{\infty} g(x)\,e^{-i\,k\,x}\,dx }
\end{equation}
An example of such a Fourier transform pair which respects this symmetrical definition
  is: if $\mathbf{g(x) = a\,e^{-b\,x^{2}}}$, then $\mathbf{G(k) = (a/\sqrt{2\,b})\,e^{-k^{2}/(4\,b)}}$. 
\begin{myVerbatim}
(%i21) assume(a>0,b>0,k>0)$
(%i22) g :a*exp(-b*x^2)$
(%i23) gft:integrate(exp(-%i*k*x)*g,x,minf,inf)/sqrt(2*%pi);
(%o23) a*%e^-(k^2/(4*b))/(sqrt(2)*sqrt(b))
\end{myVerbatim} 
\newpage
To check this relation with \mv+quag_qagi+, let's define \textbf{f} to be 
  \textbf{g} for the case \textbf{a,\,b} and \textbf{k} are all set equal to \textbf{1}. 
\begin{myVerbatim}
(%i24) f : subst([a=1,b=1,k=1],g);
(%o24) %e^-x^2
(%i25) fft : subst([a=1,b=1,k=1],gft);
(%o25) %e^-(1/4)/sqrt(2)
(%i26) float(fft);
(%o26) 0.550695
\end{myVerbatim} 
If we try to submit an explicitly complex integrand to \mv+quad_qagi+ we get
  a noun form back, indicating failure.
(A similar result occurs with any quadpack function.). 
\begin{myVerbatim}
(%i27) quad_qagi(f*exp(-%i*x),x,minf,inf);
(%o27) quad_qagi(%e^(-x^2-%i*x),x,minf,inf,epsrel = 1.0E-8,epsabs = 0.0,limit
                                                                          = 200)
\end{myVerbatim}
You can use \mv|quad_qagi| for numerical double integrals, following the pattern
  we used for \mv|quad_qags|, and you can choose different methods for each of
  the axes.
%\ remove qag, add qagp
\subsection{  Numerical Integration: Sharper Tools}
There are specialised Quadpack routines for particular kinds of
  one dimensional integrals.
\tcbr{See Sec.(\ref{finite}) for a ``decision tree'' for a finite region numerical
  integral using the Quadpack functions}.
\subsubsection{\textbf{quad\_qagp} for Internal Integrand Singularities}
The quadpack function \mv|quad_qagp| is both more efficient and more accurate 
  than \mv|quad_qags| for the
  case of integrand singularities interior to the integration interval.
The syntax is the same as \mv|quad_qags|, except that the fifth argument
  should be a list of either one or more points where the interior 
  singularities occur.
The elements of the points list should evaluate to numbers.\\

\noindent We compare the use of \mv|quad_qagp| and \mv|quad_qags| for the
numerical value of the integral
\begin{equation}
\mathbf{\int_{0}^{3} x^{3} \, \boldsymbol{\ln} \left( \, \left \vert 
         \left(x^{2} - 1 \right) \, \left(x^{2} - 2 \right) \right \vert \, \right) \, dx}
\end{equation}
which has internal singularities of the integrand at $\mathbf{x = 1}$ and
  $\mathbf{x = \boldsymbol{\sqrt{2}}}$.
\begin{myVerbatim}
(%i28) quad_qagp(x^3*log(abs((x^2-1)*(x^2-2))),x,0,3,[1,sqrt(2)]);
(%o28) [52.740748,2.62476263E-7,1029,0]
(%i29) quad_qags(x^3*log(abs((x^2-1)*(x^2-2))),x,0,3);
(%o29) [52.740748,4.08844336E-7,1869,0]
\end{myVerbatim}
\subsubsection{\textbf{quad\_qawo} for Fourier Series Coefficients}  \label{qawo}
This function is the most efficient way to find numerical values of Fourier
  series coefficients, which require finding integrals  
  $\mathbf{\int_a^b f(x)\,\boldsymbol{\cos}(\boldsymbol{\omega} \,x)\,dx}$
  or $\mathbf{\int_a^b f(x)\,\boldsymbol{\sin}(\boldsymbol{\omega} \,x)\,dx}$.\\
  
\noindent This function has the syntax
\begin{myVerbatim2}
 quad_qawo(expr, var , a, b, omega, trig,[epsrel, epsabs, limit, maxp1])
\end{myVerbatim2}  
For the integral 
  $\mathbf{\int_{-2}^{2} (x + x^{4}) \,\boldsymbol{\cos}(3 \,x)\,dx}$
  (default options), use \mvf|quad_qawo(x + x^4, x, -2, 2, 3, cos)|.\\
  
\noindent For the integral   $\mathbf{\int_{-2}^{2} (x + x^{4}) \,\boldsymbol{\sin}(5 \,x)\,dx}$
  (default options), use \mvf|quad_qawo(x + x^4, x, -2, 2, 5, sin)|.
Here we compare \mv+quad_qawo+ with \mv+quad_qags+ for the integral
  $\mathbf{\int_{-2}^{2} (x + x^{4}) \,\boldsymbol{\cos}(3 \,x)\,dx}$. 
\begin{myVerbatim}
(%i1) fpprintprec:8$
(%i2) g : (x+x^4)*cos(3*x)$
(%i3) tval : bfloat( integrate(g,x,-2,2) ),fpprec:20;
(%o3)                             3.6477501b0
(%i4) quad_qawo(x+x^4,x,-2,2,3,cos);
(%o4)                       [3.6477502, 0.0, 25, 0]
(%i5) abs(first(%) - tval),fpprec:20;
(%o5)                            4.0522056b-18
(%i6) %/tval,fpprec:20;
(%o6)                            1.110878b-18
(%i7) quad_qags(g,x,-2,2); 
(%o7)                 [3.6477502, 8.89609255E-14, 63, 0]
(%i8) abs(first(%) - tval),fpprec:20;
(%o8)                            1.7723046b-15
(%i9) %/tval,fpprec:20;
(%o9)                            4.8586239b-16
\end{myVerbatim} 
We see that \mv+quad_qawo+ finds the numerical value with ``zero'' relative error
  as compared with a ``non-zero'' relative error using \mv+quad_qags+, and with many less
  integrand evaluations.
\subsubsection{\textbf{quad\_qaws} for End Point Algebraic and Logarithmic Singularities} \label{qaws}
The syntax is:
\begin{myVerbatim2}
   quad_qaws (f(x), x, a, b, alpha, beta, wfun,[epsrel, epsabs, limit])
\end{myVerbatim2}
This Maxima function is designed for the efficient evaluation of integrals
  of the form $\mathbf{\int_{a}^{b} f(x)\, w(x)\,dx}$ in which the appropriate ``singular 
  end point weight function'' is chosen from among different versions via 
  the three parameters \textbf{wfun}, $\boldsymbol{\alpha}$ (represented 
  by \textbf{alpha}), and $\boldsymbol{\beta}$ (represented by \textbf{beta}).\\

\noindent The most general case in which one has both algebraic and logarithmic singularities of
  the integrand at both end points corresponds to  $\mathbf{\boldsymbol{\alpha} \neq 0}$
  and $\mathbf{\boldsymbol{\beta} \neq 0}$ and 
\begin{equation}
\mathbf{w(x) = (x-a)^{\boldsymbol{\alpha}}\,(b-x)^{\boldsymbol{\beta}} \,\ln(x-a)\, \ln(b-x) }
\end{equation}  
The parameters $\boldsymbol{\alpha}$ and $\boldsymbol{\beta}$ govern the ``degree''
  of algebraic singularity at the end points.
One needs both $\mathbf{\boldsymbol{\alpha} > -1}$ and $\mathbf{\boldsymbol{\beta} > -1}$ for
   convergence of the integrals.\\   

\noindent In particular, one can choose $\mathbf{\boldsymbol{\alpha} = 0}$ and/or 
  $\mathbf{\boldsymbol{\beta} =0}$
  to handle an algebraic singularity at only one end of the interval or
  no algebraic singularities at all.\\
  
\noindent The parameter \textbf{wfun} determines the existence and location of possible
  end point logarithmic singularities of the integrand.  
\begin{center}
  \begin{tabular}{cc}
    wfun & w(x) \\
	\hline
	1 & $\mathbf{(x-a)^{\boldsymbol{\alpha}}\,(b-x)^{\boldsymbol{\beta}}}$ \\
	2 & $\mathbf{(x-a)^{\boldsymbol{\alpha}}\,(b-x)^{\boldsymbol{\beta}} \,\boldsymbol{\ln}(x-a)}$ \\
	3 & $\mathbf{(x-a)^{\boldsymbol{\alpha}}\,(b-x)^{\boldsymbol{\beta}} \, \boldsymbol{\ln}(b-x)}$ \\
	4 & $\mathbf{(x-a)^{\boldsymbol{\alpha}}\,(b-x)^{\boldsymbol{\beta}} \,\boldsymbol{\ln(x-a)\, \ln(b-x)}}$ \\ 
   \end{tabular}
\end{center}
\subsubsection*{Example 1: Pure Logarithmic Singularities}
For the case that $\mathbf{\boldsymbol{\alpha} = 0 }$ 
   and $\mathbf{\boldsymbol{\beta} = 0}$, there are
  no end point algebraic singularities, only logarithmic singularities.
A simple example is $\mathbf{\int_{0}^{1} \boldsymbol{\ln}(x) \, dx}$, which 
   corresponds to $\mathbf{wfun = 2}$: 
\begin{myVerbatim}
(%i1) fpprintprec:8$
(%i2) tval : bfloat( integrate(log(x),x,0,1) ),fpprec:20;
(%o2)                               - 1.0b0
(%i3) quad_qaws(1,x,0,1,0,0,2);
(%o3)                   [- 1.0, 9.68809031E-15, 40, 0]
(%i4) abs(first(%) - tval),fpprec:20;
(%o4)                                0.0b0
(%i5) quad_qags(log(x),x,0,1);
(%o5)                   [- 1.0, 1.11022302E-15, 231, 0]
(%i6) abs(first(%) - tval),fpprec:20;
(%o6)                                0.0b0
\end{myVerbatim} 
which illustrates the efficiency of \mv+quad_qaws+ compared to \mv+quad_qags+
  for this type of integrand.
\subsubsection*{Example 2: Pure Algebraic Singularity }
The case $\mathbf{wfun} = 1$ corresponds to purely algebraic
  end point singularities.  \\

\noindent Here we compare \mv+quad_qaws+ with \mv+quad_qags+ for
  the evaluation of the integral $\mathbf{\int_{0}^{1} \frac{\boldsymbol{\sin}(x)}{\sqrt{x}}\, dx}$.
You will get an exact symbolic answer in terms of \textbf{erf(z)} for this 
  integral from \textbf{integrate}.
% check bfloat erf(z) with next version of maxima : ver 18.1 doesn't cure
% the problem of getting everything numerical.
\begin{myVerbatim}
(%i7) expand(bfloat(integrate(sin(x)/sqrt(x),x,0,1))),fpprec:20;
(%o7)       1.717976b0 cos(0.25 %pi) - 8.404048b-1 sin(0.25 %pi)
(%i8) tval : bfloat(%),fpprec:20;
(%o8)                            6.205366b-1
(%i9) quad_qaws(sin(x),x,0,1,-1/2,0,1);
(%o9)                 [0.620537, 4.31887834E-15, 40, 0]
(%i10) abs(first(%) - tval),fpprec:20;
(%o10)                           8.8091426b-19
(%i11) %/tval,fpprec:20;
(%o11)                           1.4196008b-18
(%i12) quad_qags(sin(x)/sqrt(x),x,0,1);
(%o12)                [0.620537, 3.48387985E-13, 231, 0]
(%i13) abs(first(%) - tval),fpprec:20;
(%o13)                           1.1014138b-16
(%i14) %/tval,fpprec:20;
(%o14)                           1.7749378b-16
\end{myVerbatim} 
We see that \mv+quad_qaws+ uses about one sixth the number of function evaluations
  (as compared with \mv+quad_qags+) and returns a much more accurate answer.
\subsubsection*{Example 3: Both Algebraic and Logarithmic Singularity at an End Point }
A simple example is $\mathbf{\int_{0}^{1} \frac{ \boldsymbol{\ln(x)}} {\sqrt{x} }\, dx}$.
The integrand is singular at $\mathbf{x = 0}$ but this is an ``integrable singularity''
  since $\mathbf{\sqrt{x} \; \boldsymbol{\ln}(x) \rightarrow 0}$ 
   as $\mathbf{x \rightarrow 0^{+}}$.
\begin{myVerbatim}
(%i1) fpprintprec:8$
(%i2) limit(log(x)/sqrt(x),x,0,plus);
(%o2)                                minf
\end{myVerbatim}
\newpage
\begin{myVerbatim}
(%i3) integrate(log(x)/sqrt(x),x);
(%o3)                   2 (sqrt(x) log(x) - 2 sqrt(x))
(%i4) limit(%,x,0,plus);
(%o4)                                  0
(%i5) tval : bfloat( integrate(log(x)/sqrt(x),x,0,1)),fpprec:20;
(%o5)                               - 4.0b0
(%i6) quad_qaws(1,x,0,1,-1/2,0,2);
(%o6)                   [- 4.0, 3.59396672E-13, 40, 0]
(%i7) abs(first(%) - tval),fpprec:20;
(%o7)                                0.0b0
(%i8) quad_qags(log(x)/sqrt(x),x,0,1);
(%o8)                   [- 4.0, 1.94066985E-13, 315, 0]
(%i9) abs(first(%) - tval),fpprec:20;
(%o9)                            2.6645352b-15
(%i10) %/tval,fpprec:20;
(%o10)                          - 6.6613381b-16
\end{myVerbatim} 
Again we see the relative efficiency and accuracy of \mv+quad_qaws+ for this type
  of integral.
\subsubsection{\textbf{quad\_qawc} for a Cauchy Principal Value Integral}
This function has the syntax: 
\begin{myVerbatim2}
        quad_qawc (f(x), x, c, a, b,[epsrel, epsabs, limit]). 
\end{myVerbatim2}
\noindent The actual integrand is $\mathbf{g(x) = f(x)/(x - c)}$, 
  with dependent variable \textbf{x}, to be integrated over 
  the interval $\mathbf{[a, b]}$ and you need to pick out $\mathbf{f(x)}$
  by hand here.
In using \mv+quad_qawc+, the argument $\mathbf{c}$ is placed between  
   the name of the variable of integration (here \textbf{x}) and
   the lower limit of integration.\\

\noindent An integral with a ``pole'' on the contour does not exist in the
  strict sense, but if $\mathbf{g(x)}$ has a simple pole on the real axis
   at $\mathbf{x = c }$, one defines the
  Cauchy principal value as the symmetrical limit (with $\mathbf{a < c < b}$)
\begin{equation}  \label{Eq:pval}
\mathbf{P \,\int_{a}^{b} g(x)\, dx = 
   \lim_{\boldsymbol{\epsilon} \rightarrow 0^{+}} 
   \left[ \int_{a}^{c - \boldsymbol{\epsilon}} g(x) \, dx + 
           \int_{c + \boldsymbol{\epsilon}}^{b} g(x) \, dx \right] }
\end{equation}  
  provided this limit exists.
In terms of $\mathbf{f(x)}$ this definition becomes  
\begin{equation}
\mathbf{P \,\int_{a}^{b} \frac{f(x)}{x - c} \, dx = 
   \lim_{\boldsymbol{\epsilon} \rightarrow 0^{+}} 
   \left[ \int_{a}^{c - \boldsymbol{\epsilon}} \frac{f(x)}{x - c} \, dx + 
           \int_{c + \boldsymbol{\epsilon}}^{b} \frac{f(x)}{x - c} \, dx \right] }
\end{equation}
%\newpage
We can find the default values of the optional method parameters of \mv+quad_qawc+ by
  including an undefined symbol in our call:
\begin{myVerbatim}
(%i11) quad_qawc(1/(x^2-1),x,1,0,b);
                  1
(%o11) quad_qawc(------, x, 1, 0, b, epsrel = 1.0E-8, epsabs = 0.0, limit = 200)
                 2
                x  - 1
\end{myVerbatim}
We see that the default settings cause the algorithm to look at the relative 
  error of succeeding approximations to the numerical answer.\\

\noindent As a simple example of the syntax we consider the principal value integral
\begin{equation}
\mathbf{P \, \int_{0}^{2} \frac{1}{x^{2} -1} \,dx = 
          P \, \int_{0}^{2} \frac{1}{(x-1)\,(x+1)} \,dx = - \boldsymbol{\ln}(3)/2 }
\end{equation}
We use \textbf{assume} to prep \textbf{integrate} and then implement
  the basic definition provided by Eq. (\ref{Eq:pval})	
\begin{myVerbatim}
(%i1) fpprintprec:8$
(%i2) assume(eps>0, eps<1)$
(%i3) integrate(1/(x^2-1),x,0,1-eps) +
               integrate(1/(x^2-1),x,1+eps,2);
                     log(eps + 2)   log(2 - eps)   log(3)
(%o3)                ------------ - ------------ - ------
                          2              2           2
(%i4) limit(%,eps,0,plus);
                                     log(3)
(%o4)                              - ------
                                       2
(%i5) tval : bfloat(%),fpprec:20;
(%o5)                           - 5.4930614b-1
\end{myVerbatim}
We now compare the result returned by \textbf{integrate} with the numerical
  value returned by \mv+quad_qawc+, noting that $\mathbf{f(x) = 1/(1+x)}$.
\begin{myVerbatim}
(%i6) quad_qawc(1/(1+x),x,1,0,2);
(%o6)                [- 0.549306, 1.51336373E-11, 105, 0]
(%i7) abs(first(%) - tval),fpprec:20;
(%o7)                            6.5665382b-17
(%i8) %/tval,fpprec:20;
(%o8)                           - 1.1954241b-16
\end{myVerbatim}
We see that the relative error of the returned answer is much less than
  the (default) requested minimum relative error.\\
  
\noindent If we run \mv+quad_qawc+ requesting that convergence be based
  on absolute error instead of relative error,
\begin{myVerbatim}
(%i9) quad_qawc(1/(1+x),x,1,0,2,epsabs=1.0e-10,epsrel=0.0);
(%o9)                [- 0.549306, 1.51336373E-11, 105, 0]
(%i10) abs(first(%) - tval),fpprec:20;
(%o10)                           6.5665382b-17
(%i11) %/tval,fpprec:20;
(%o11)                          - 1.1954241b-16
\end{myVerbatim}
we see no significant difference in the returned accuracy, and again we
  see that the absolute error of the returned answer is much less than
  the requested minimum absolute error.  
\newpage
\subsubsection{\textbf{quad\_qawf} for a Semi-Infinite Range Cosine or Sine Fourier Transform}
The function \mv+quad_qawf+ calculates a Fourier cosine \emph{or} 
   Fourier sine transform (up to an overall normalization factor) on the 
   semi-infinite interval $\mathbf{[a, \infty]}$.
If we let \textbf{w} stand for the angular frequency in radians,
  the integrand is \textbf{f(x)*cos(w*x)} if the \textbf{trig} parameter is \textbf{cos},
  and the integrand is \textbf{f(x)*sin(w*x)} if the \textbf{trig} parameter is \textbf{sin}.\\

\noindent The calling syntax is 
\begin{myVerbatim2}
     quad_qawf (f(x), x, a, w, trig, [epsabs, limit, maxp1, limlst]) 
\end{myVerbatim2}
Thus \mvs|quad_qawf (f(x), x, 0, w, 'cos)| will find a numerical approximation
  to the integral
\begin{equation}
\mathbf{ \int_{0}^{\infty} f(x)\, \boldsymbol{\cos} (w \, x) \, dx }
\end{equation}
If we call \mv+quad_qawf+ with undefined parameter(s), we get a look at the
  default values of the optional method parameters:
\begin{myVerbatim}
(%i12) quad_qawf(exp(-a*x),x,0,w,'cos);
                  - a x
(%o12) quad_qawf(%e     , x, 0, w, cos, epsabs = 1.0E-10, limit = 200, 
                                                      maxp1 = 100, limlst = 10)
\end{myVerbatim}
%\newpage
\noindent The manual has the optional parameter information:
\begin{myVerbatim2f}
The keyword arguments are optional and may be specified in any order.
They all take the form keyword = val. The keyword arguments are: 
1. epsabs, the desired absolute error of approximation. Default is 1d-10. 
2. limit, the size of the internal work array. 
   (limit - limlst)/2 is the maximum number of 
   subintervals to use. The default value of limit is 200. 
3. maxp1, the maximum number of Chebyshev moments. 
      Must be greater than 0. Default is 100. 
4. limlst, upper bound on the number of cycles. 
    Must be greater than or equal to 3. Default is 10. 
\end{myVerbatim2f}
The manual does not define the meaning of ``cycles''.
There is no ``epsrel'' parameter used for this function.\\

\noindent Here is the manual example, organised in our way.
In this example $\mathbf{w = 1}$ and $\mathbf{a = 0}$.
\begin{myVerbatim}
(%i1) fpprintprec:8$
(%i2) integrate (exp(-x^2)*cos(x), x, 0, inf);
                                 - 1/4
                               %e      sqrt(%pi)
(%o2)                          -----------------
                                       2
(%i3) tval : bfloat(%),fpprec:20;
(%o3)                            6.9019422b-1
(%i4) quad_qawf (exp(-x^2), x, 0, 1, 'cos );
(%o4)                 [0.690194, 2.84846299E-11, 215, 0]
(%i5) abs(first(%) - tval),fpprec:20;
(%o5)                            3.909904b-17
(%i6) %/tval,fpprec:20;
(%o6)                            5.664933b-17
\end{myVerbatim}
We see that the absolute error of the returned answer is much less than
  the (default) requested minimum absolute error.
% from raymond toy:
%There is no limlst for qag and
%qags.  So limit defaults to 200 and is the maximum number of subintervals.
%limlst appears to be only used by qawf.
\subsection{   Finite Region of Integration Decision Tree}  \label{finite}
If you are in a hurry, use \mv+quad_qags+.
\footnote{ These ``decision trees'' are adapted from the web page\\
  \mvf|http://people.scs.fsu.edu/$\sim$burkardt/f77\_src/quadpack/quadpack.html|.} \\
 
\noindent \tcbr{If your definite integral is over a finite region of integration}
    \textbf{[a, b]}, then%
\begin{enumerate}
   \item If the integrand has the form \textbf{w(x)*f(x)}, where \textbf{f(x)} is a smooth function
           over the region of integration, then
	\begin{itemize}
        \item 	If  \textbf{w(x)} has the form of either \textbf{ cos( c*x )} 
		or \textbf{sin( c*x )}, where \textbf{c} is a constant, use \mv+quad_qawo+.
		\item  Else if the factor \textbf{w(x)} has the form (note the limits of integration
		  are \textbf{[ a, b ]}
\begin{myVerbatim2s}
    (x - a)^ae * (b - x)^be * (log(x - a))^na * (log(b - x))^nb
\end{myVerbatim2s}
  where \textbf{na}, \textbf{nb} have the values \textbf{0} or \textbf{1}, 
   and both \textbf{ae} and \textbf{be} are greater than $\mathbf{-1}$, 
   then use \mv+quad_qaws+.		 
(This is a case where we need a routine which is designed to
		                handle  end point singularities.)
		\item Else if the factor \textbf{w(x)} is \textbf{1/(x - c)} for 
		some constant \textbf{c} with $\mathbf{a < c < b}$, then 
		use \mv+quad_qawc+, the Cauchy principle value
		routine.
       \end{itemize}
    \item 	Otherwise, if you do not care too much about possible inefficient
       use of computer time, and do not want to further analyze the
	   problem, use \mv+quad_qags+. 
    \item Otherwise, if the integrand is \textbf{smooth}, use \mv+quad_qag+.
	\item Otherwise, if there are \textbf{discontinuities or singularities} of the integrand 
      or of the derivative of the integrand, and you know where they are,
	  split the integration range at these points and separately integrate
      over each subinterval.
    \item Otherwise, if the integrand has \textbf{end point singularities}, 
	  use \mv+quad_qags+. 
    \item Otherwise, if the integrand has an oscillatory behavior of nonspecific type,
         and no singularities, use \mv+quad_qag+ with 
		 the fifth \textbf{key} slot containing the value \textbf{6}. 	   
	\item 	 Otherwise, use \mv+quad_qags+. 
\end{enumerate} 
\newpage
\subsection{Non-Finite Region of Integration Decision Tree } \label{inf}   
Here is the tree for the non-finite domain case:
\begin{myVerbatim2s} 
A.  If the integration domain is non-finite consider first making a
       suitable change of variables to obtain a finite domain and 
	   using the finite interval decision tree described above.

B.  Otherwise, if the integrand decays rapidly to zero, truncate the
      integration interval and use the finite interval decision tree. 
	  
C.  Otherwise, if the integrand oscillates over the entire non-finite range,

       1. If integral is a Fourier transform, use quad_qawf. 

       2. else if the integral is not a Fourier transform, then sum the
           successive positive and negative contributions by integrating
           between the zeroes of the integrand, using the finite 
           interval criteria.
  
D.  Otherwise, if you are not constrained by computer time, and do not
     wish to analyze the problem further, use quad_qagi. 

E.  Otherwise, if the integrand has a non-smooth behavior in the range of
     integration, and you know where it occurs, split off these regions
     and use the appropriate finite range routines to integrate over them.
     Then begin this non-finite case tree again to handle the remainder
	 of the region. 

F.  Otherwise, truncation of the interval, or application of a suitable
     transformation for reducing the problem to a finite range may be 
     possible.
		  
G.  Otherwise use quad_qagi. 
\end{myVerbatim2s}
\end{document}